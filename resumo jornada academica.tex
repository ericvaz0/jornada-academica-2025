\documentclass[a4paper,12pt]{article}
\usepackage[brazil]{babel} 
\usepackage[utf8]{inputenc}

\usepackage[letterpaper,top=2cm,bottom=2cm,left=3cm,right=3cm,marginparwidth=1.75cm]{geometry}
% Useful packages
\usepackage{amsmath}
\usepackage{graphicx}
\usepackage[colorlinks=true, allcolors=blue]{hyperref}

\title{Grupos Livres e o Lema do Ping-Pong\footnote{Trabalho submetido para a XVI Escola de Verão em Matemática da UFS na sessão V Jornada acadêmica.}}
\author{Aluno: Eric Vaz de Souza Santos\\ Orientador: Prof. Dr. Cayo Rodrigo Felizardo Dória\\
Instituição: Universidade Federal de Sergipe\\
e-mail: \texttt{ericvaz1406@gmail.com}\\
Nível: Mestrado}

\begin{document}
\maketitle

\begin{abstract}
    O Lema do Ping-Pong é um resultado central na teoria geométrica dos grupos, fornecendo uma condição suficiente para a existência de subgrupos livres. Discutiremos situações em que a presença desses subgrupos livres tem consequências interessantes. Um exemplo é a Alternativa de Tits, um teorema fundamental sobre a estrutura de grupos lineares finitamente gerados, que mostra que esses grupos ou contêm subgrupos livres ou são virtualmente solúveis. Também mencionaremos o Paradoxo de Banach-Tarski, um resultado surpreendente da teoria da medida que depende da existência de subgrupos livres em grupos de rotação.
\end{abstract}

\end{document}